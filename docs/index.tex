% Options for packages loaded elsewhere
\PassOptionsToPackage{unicode}{hyperref}
\PassOptionsToPackage{hyphens}{url}
%
\documentclass[
]{article}
\usepackage{amsmath,amssymb}
\usepackage{lmodern}
\usepackage{ifxetex,ifluatex}
\ifnum 0\ifxetex 1\fi\ifluatex 1\fi=0 % if pdftex
  \usepackage[T1]{fontenc}
  \usepackage[utf8]{inputenc}
  \usepackage{textcomp} % provide euro and other symbols
\else % if luatex or xetex
  \usepackage{unicode-math}
  \defaultfontfeatures{Scale=MatchLowercase}
  \defaultfontfeatures[\rmfamily]{Ligatures=TeX,Scale=1}
\fi
% Use upquote if available, for straight quotes in verbatim environments
\IfFileExists{upquote.sty}{\usepackage{upquote}}{}
\IfFileExists{microtype.sty}{% use microtype if available
  \usepackage[]{microtype}
  \UseMicrotypeSet[protrusion]{basicmath} % disable protrusion for tt fonts
}{}
\makeatletter
\@ifundefined{KOMAClassName}{% if non-KOMA class
  \IfFileExists{parskip.sty}{%
    \usepackage{parskip}
  }{% else
    \setlength{\parindent}{0pt}
    \setlength{\parskip}{6pt plus 2pt minus 1pt}}
}{% if KOMA class
  \KOMAoptions{parskip=half}}
\makeatother
\usepackage{xcolor}
\IfFileExists{xurl.sty}{\usepackage{xurl}}{} % add URL line breaks if available
\IfFileExists{bookmark.sty}{\usepackage{bookmark}}{\usepackage{hyperref}}
\hypersetup{
  pdftitle={Untitled},
  pdfauthor={Lucia Briganti},
  hidelinks,
  pdfcreator={LaTeX via pandoc}}
\urlstyle{same} % disable monospaced font for URLs
\usepackage[margin=1in]{geometry}
\usepackage{color}
\usepackage{fancyvrb}
\newcommand{\VerbBar}{|}
\newcommand{\VERB}{\Verb[commandchars=\\\{\}]}
\DefineVerbatimEnvironment{Highlighting}{Verbatim}{commandchars=\\\{\}}
% Add ',fontsize=\small' for more characters per line
\usepackage{framed}
\definecolor{shadecolor}{RGB}{248,248,248}
\newenvironment{Shaded}{\begin{snugshade}}{\end{snugshade}}
\newcommand{\AlertTok}[1]{\textcolor[rgb]{0.94,0.16,0.16}{#1}}
\newcommand{\AnnotationTok}[1]{\textcolor[rgb]{0.56,0.35,0.01}{\textbf{\textit{#1}}}}
\newcommand{\AttributeTok}[1]{\textcolor[rgb]{0.77,0.63,0.00}{#1}}
\newcommand{\BaseNTok}[1]{\textcolor[rgb]{0.00,0.00,0.81}{#1}}
\newcommand{\BuiltInTok}[1]{#1}
\newcommand{\CharTok}[1]{\textcolor[rgb]{0.31,0.60,0.02}{#1}}
\newcommand{\CommentTok}[1]{\textcolor[rgb]{0.56,0.35,0.01}{\textit{#1}}}
\newcommand{\CommentVarTok}[1]{\textcolor[rgb]{0.56,0.35,0.01}{\textbf{\textit{#1}}}}
\newcommand{\ConstantTok}[1]{\textcolor[rgb]{0.00,0.00,0.00}{#1}}
\newcommand{\ControlFlowTok}[1]{\textcolor[rgb]{0.13,0.29,0.53}{\textbf{#1}}}
\newcommand{\DataTypeTok}[1]{\textcolor[rgb]{0.13,0.29,0.53}{#1}}
\newcommand{\DecValTok}[1]{\textcolor[rgb]{0.00,0.00,0.81}{#1}}
\newcommand{\DocumentationTok}[1]{\textcolor[rgb]{0.56,0.35,0.01}{\textbf{\textit{#1}}}}
\newcommand{\ErrorTok}[1]{\textcolor[rgb]{0.64,0.00,0.00}{\textbf{#1}}}
\newcommand{\ExtensionTok}[1]{#1}
\newcommand{\FloatTok}[1]{\textcolor[rgb]{0.00,0.00,0.81}{#1}}
\newcommand{\FunctionTok}[1]{\textcolor[rgb]{0.00,0.00,0.00}{#1}}
\newcommand{\ImportTok}[1]{#1}
\newcommand{\InformationTok}[1]{\textcolor[rgb]{0.56,0.35,0.01}{\textbf{\textit{#1}}}}
\newcommand{\KeywordTok}[1]{\textcolor[rgb]{0.13,0.29,0.53}{\textbf{#1}}}
\newcommand{\NormalTok}[1]{#1}
\newcommand{\OperatorTok}[1]{\textcolor[rgb]{0.81,0.36,0.00}{\textbf{#1}}}
\newcommand{\OtherTok}[1]{\textcolor[rgb]{0.56,0.35,0.01}{#1}}
\newcommand{\PreprocessorTok}[1]{\textcolor[rgb]{0.56,0.35,0.01}{\textit{#1}}}
\newcommand{\RegionMarkerTok}[1]{#1}
\newcommand{\SpecialCharTok}[1]{\textcolor[rgb]{0.00,0.00,0.00}{#1}}
\newcommand{\SpecialStringTok}[1]{\textcolor[rgb]{0.31,0.60,0.02}{#1}}
\newcommand{\StringTok}[1]{\textcolor[rgb]{0.31,0.60,0.02}{#1}}
\newcommand{\VariableTok}[1]{\textcolor[rgb]{0.00,0.00,0.00}{#1}}
\newcommand{\VerbatimStringTok}[1]{\textcolor[rgb]{0.31,0.60,0.02}{#1}}
\newcommand{\WarningTok}[1]{\textcolor[rgb]{0.56,0.35,0.01}{\textbf{\textit{#1}}}}
\usepackage{graphicx}
\makeatletter
\def\maxwidth{\ifdim\Gin@nat@width>\linewidth\linewidth\else\Gin@nat@width\fi}
\def\maxheight{\ifdim\Gin@nat@height>\textheight\textheight\else\Gin@nat@height\fi}
\makeatother
% Scale images if necessary, so that they will not overflow the page
% margins by default, and it is still possible to overwrite the defaults
% using explicit options in \includegraphics[width, height, ...]{}
\setkeys{Gin}{width=\maxwidth,height=\maxheight,keepaspectratio}
% Set default figure placement to htbp
\makeatletter
\def\fps@figure{htbp}
\makeatother
\setlength{\emergencystretch}{3em} % prevent overfull lines
\providecommand{\tightlist}{%
  \setlength{\itemsep}{0pt}\setlength{\parskip}{0pt}}
\setcounter{secnumdepth}{-\maxdimen} % remove section numbering
\ifluatex
  \usepackage{selnolig}  % disable illegal ligatures
\fi

\title{Untitled}
\author{Lucia Briganti}
\date{11/7/2021}

\begin{document}
\maketitle

\hypertarget{visualize-meteorological-data}{%
\subsection{Visualize meteorological
data!}\label{visualize-meteorological-data}}

Easy codee for the visualization of wind variable and analyzing any
anomalies.

\begin{Shaded}
\begin{Highlighting}[]
\CommentTok{\# upload libraries and dataset}

\FunctionTok{library}\NormalTok{(openair)}
\FunctionTok{library}\NormalTok{(readxl)}
\end{Highlighting}
\end{Shaded}

First I Pick up the data (this file was provided by the National Weather
Service and contains information about meteorological variables from an
airport station).

\begin{Shaded}
\begin{Highlighting}[]
\NormalTok{myData }\OtherTok{\textless{}{-}} \FunctionTok{read\_excel}\NormalTok{(}\StringTok{"C:/Users/Lucia/Desktop/TESIS/Datos/data meteo/dataint\_meteo\_2010\_2020.xlsx"}\NormalTok{)}
\NormalTok{myData }\OtherTok{\textless{}{-}} \FunctionTok{na.omit}\NormalTok{(myData)}
\FunctionTok{head}\NormalTok{(myData)}
\end{Highlighting}
\end{Shaded}

\begin{verbatim}
## # A tibble: 6 x 9
##    year month   day  hour    AT    RH    SC    WD    WS
##   <dbl> <dbl> <dbl> <dbl> <dbl> <dbl> <dbl> <dbl> <dbl>
## 1  2010     5     1     0  18    70.6     2   320  6.17
## 2  2010     5     1     1  18.3  64.4     4   320  4.63
## 3  2010     5     1     2  18.2  65.2     4   320  5.14
## 4  2010     5     1     3  18    66.1     0   320  5.66
## 5  2010     5     1     4  17    62.7     0   320  5.66
## 6  2010     5     1     5  17    60.7     0   320  5.66
\end{verbatim}

I have hourly data, so I will add a date column and round WS (wind
speed) digits, this will be useful for the wind rose.

\begin{Shaded}
\begin{Highlighting}[]
\NormalTok{myData}\SpecialCharTok{$}\NormalTok{date }\OtherTok{\textless{}{-}} \FunctionTok{as.Date}\NormalTok{(}\FunctionTok{with}\NormalTok{(myData, }\FunctionTok{paste}\NormalTok{(year, month, day,}\AttributeTok{sep=}\StringTok{"{-}"}\NormalTok{)), }\StringTok{"\%Y{-}\%m{-}\%d"}\NormalTok{)}
\NormalTok{myData}\SpecialCharTok{$}\NormalTok{WS }\OtherTok{\textless{}{-}} \FunctionTok{round}\NormalTok{(myData}\SpecialCharTok{$}\NormalTok{WS, }\AttributeTok{digits =} \DecValTok{2}\NormalTok{)}
\FunctionTok{head}\NormalTok{(myData)}
\end{Highlighting}
\end{Shaded}

\begin{verbatim}
## # A tibble: 6 x 10
##    year month   day  hour    AT    RH    SC    WD    WS date      
##   <dbl> <dbl> <dbl> <dbl> <dbl> <dbl> <dbl> <dbl> <dbl> <date>    
## 1  2010     5     1     0  18    70.6     2   320  6.17 2010-05-01
## 2  2010     5     1     1  18.3  64.4     4   320  4.63 2010-05-01
## 3  2010     5     1     2  18.2  65.2     4   320  5.14 2010-05-01
## 4  2010     5     1     3  18    66.1     0   320  5.66 2010-05-01
## 5  2010     5     1     4  17    62.7     0   320  5.66 2010-05-01
## 6  2010     5     1     5  17    60.7     0   320  5.66 2010-05-01
\end{verbatim}

I want to visualize the historical wind pattern during winter in the
city of Buenos Aires and compare it with the 2020 winter wind pattern,
so for that I am plotting a wind rose for the last ten years (2010-2019)
and another one for 2020. So I just subset up WS (wind speed) and WD
(wind direction) from the dataframe. windRose() is a function of the
openair package, very easy to use and aesthetic too.

\begin{Shaded}
\begin{Highlighting}[]
\FunctionTok{plot}\NormalTok{(WR\_2020)}
\end{Highlighting}
\end{Shaded}

\includegraphics{C:/Users/Lucia/Desktop/GITHUB/Visualizing-met-data/docs/index_files/figure-latex/unnamed-chunk-5-1.pdf}

\begin{Shaded}
\begin{Highlighting}[]
\FunctionTok{plot}\NormalTok{(WR\_climatica)}
\end{Highlighting}
\end{Shaded}

\includegraphics{C:/Users/Lucia/Desktop/GITHUB/Visualizing-met-data/docs/index_files/figure-latex/unnamed-chunk-5-2.pdf}

This is a qualitative analyse very useful when you are looking for wind
anomalies in the wind and wanna do it quickly, it worked for me in my
thesis and when analyzing a particular event. I truly recommend it:
checking the event with the climatology is always a good idea. Hope it
was useful =)

\end{document}
